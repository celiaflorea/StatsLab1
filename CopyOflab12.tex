% Options for packages loaded elsewhere
\PassOptionsToPackage{unicode}{hyperref}
\PassOptionsToPackage{hyphens}{url}
%
\documentclass[
  english,
  man]{article}
\usepackage{amsmath,amssymb}
\usepackage{lmodern}
\usepackage{ifxetex,ifluatex}
\ifnum 0\ifxetex 1\fi\ifluatex 1\fi=0 % if pdftex
  \usepackage[T1]{fontenc}
  \usepackage[utf8]{inputenc}
  \usepackage{textcomp} % provide euro and other symbols
\else % if luatex or xetex
  \usepackage{unicode-math}
  \defaultfontfeatures{Scale=MatchLowercase}
  \defaultfontfeatures[\rmfamily]{Ligatures=TeX,Scale=1}
\fi
% Use upquote if available, for straight quotes in verbatim environments
\IfFileExists{upquote.sty}{\usepackage{upquote}}{}
\IfFileExists{microtype.sty}{% use microtype if available
  \usepackage[]{microtype}
  \UseMicrotypeSet[protrusion]{basicmath} % disable protrusion for tt fonts
}{}
\makeatletter
\@ifundefined{KOMAClassName}{% if non-KOMA class
  \IfFileExists{parskip.sty}{%
    \usepackage{parskip}
  }{% else
    \setlength{\parindent}{0pt}
    \setlength{\parskip}{6pt plus 2pt minus 1pt}}
}{% if KOMA class
  \KOMAoptions{parskip=half}}
\makeatother
\usepackage{xcolor}
\IfFileExists{xurl.sty}{\usepackage{xurl}}{} % add URL line breaks if available
\IfFileExists{bookmark.sty}{\usepackage{bookmark}}{\usepackage{hyperref}}
\hypersetup{
  pdflang={en-EN},
  hidelinks,
  pdfcreator={LaTeX via pandoc}}
\urlstyle{same} % disable monospaced font for URLs
\usepackage{color}
\usepackage{fancyvrb}
\newcommand{\VerbBar}{|}
\newcommand{\VERB}{\Verb[commandchars=\\\{\}]}
\DefineVerbatimEnvironment{Highlighting}{Verbatim}{commandchars=\\\{\}}
% Add ',fontsize=\small' for more characters per line
\usepackage{framed}
\definecolor{shadecolor}{RGB}{248,248,248}
\newenvironment{Shaded}{\begin{snugshade}}{\end{snugshade}}
\newcommand{\AlertTok}[1]{\textcolor[rgb]{0.94,0.16,0.16}{#1}}
\newcommand{\AnnotationTok}[1]{\textcolor[rgb]{0.56,0.35,0.01}{\textbf{\textit{#1}}}}
\newcommand{\AttributeTok}[1]{\textcolor[rgb]{0.77,0.63,0.00}{#1}}
\newcommand{\BaseNTok}[1]{\textcolor[rgb]{0.00,0.00,0.81}{#1}}
\newcommand{\BuiltInTok}[1]{#1}
\newcommand{\CharTok}[1]{\textcolor[rgb]{0.31,0.60,0.02}{#1}}
\newcommand{\CommentTok}[1]{\textcolor[rgb]{0.56,0.35,0.01}{\textit{#1}}}
\newcommand{\CommentVarTok}[1]{\textcolor[rgb]{0.56,0.35,0.01}{\textbf{\textit{#1}}}}
\newcommand{\ConstantTok}[1]{\textcolor[rgb]{0.00,0.00,0.00}{#1}}
\newcommand{\ControlFlowTok}[1]{\textcolor[rgb]{0.13,0.29,0.53}{\textbf{#1}}}
\newcommand{\DataTypeTok}[1]{\textcolor[rgb]{0.13,0.29,0.53}{#1}}
\newcommand{\DecValTok}[1]{\textcolor[rgb]{0.00,0.00,0.81}{#1}}
\newcommand{\DocumentationTok}[1]{\textcolor[rgb]{0.56,0.35,0.01}{\textbf{\textit{#1}}}}
\newcommand{\ErrorTok}[1]{\textcolor[rgb]{0.64,0.00,0.00}{\textbf{#1}}}
\newcommand{\ExtensionTok}[1]{#1}
\newcommand{\FloatTok}[1]{\textcolor[rgb]{0.00,0.00,0.81}{#1}}
\newcommand{\FunctionTok}[1]{\textcolor[rgb]{0.00,0.00,0.00}{#1}}
\newcommand{\ImportTok}[1]{#1}
\newcommand{\InformationTok}[1]{\textcolor[rgb]{0.56,0.35,0.01}{\textbf{\textit{#1}}}}
\newcommand{\KeywordTok}[1]{\textcolor[rgb]{0.13,0.29,0.53}{\textbf{#1}}}
\newcommand{\NormalTok}[1]{#1}
\newcommand{\OperatorTok}[1]{\textcolor[rgb]{0.81,0.36,0.00}{\textbf{#1}}}
\newcommand{\OtherTok}[1]{\textcolor[rgb]{0.56,0.35,0.01}{#1}}
\newcommand{\PreprocessorTok}[1]{\textcolor[rgb]{0.56,0.35,0.01}{\textit{#1}}}
\newcommand{\RegionMarkerTok}[1]{#1}
\newcommand{\SpecialCharTok}[1]{\textcolor[rgb]{0.00,0.00,0.00}{#1}}
\newcommand{\SpecialStringTok}[1]{\textcolor[rgb]{0.31,0.60,0.02}{#1}}
\newcommand{\StringTok}[1]{\textcolor[rgb]{0.31,0.60,0.02}{#1}}
\newcommand{\VariableTok}[1]{\textcolor[rgb]{0.00,0.00,0.00}{#1}}
\newcommand{\VerbatimStringTok}[1]{\textcolor[rgb]{0.31,0.60,0.02}{#1}}
\newcommand{\WarningTok}[1]{\textcolor[rgb]{0.56,0.35,0.01}{\textbf{\textit{#1}}}}
\usepackage{graphicx}
\makeatletter
\def\maxwidth{\ifdim\Gin@nat@width>\linewidth\linewidth\else\Gin@nat@width\fi}
\def\maxheight{\ifdim\Gin@nat@height>\textheight\textheight\else\Gin@nat@height\fi}
\makeatother
% Scale images if necessary, so that they will not overflow the page
% margins by default, and it is still possible to overwrite the defaults
% using explicit options in \includegraphics[width, height, ...]{}
\setkeys{Gin}{width=\maxwidth,height=\maxheight,keepaspectratio}
% Set default figure placement to htbp
\makeatletter
\def\fps@figure{htbp}
\makeatother
\setlength{\emergencystretch}{3em} % prevent overfull lines
\providecommand{\tightlist}{%
  \setlength{\itemsep}{0pt}\setlength{\parskip}{0pt}}
\setcounter{secnumdepth}{-\maxdimen} % remove section numbering
% Make \paragraph and \subparagraph free-standing
\ifx\paragraph\undefined\else
  \let\oldparagraph\paragraph
  \renewcommand{\paragraph}[1]{\oldparagraph{#1}\mbox{}}
\fi
\ifx\subparagraph\undefined\else
  \let\oldsubparagraph\subparagraph
  \renewcommand{\subparagraph}[1]{\oldsubparagraph{#1}\mbox{}}
\fi
% Manuscript styling
\usepackage{upgreek}
\captionsetup{font=singlespacing,justification=justified}

% Table formatting
\usepackage{longtable}
\usepackage{lscape}
% \usepackage[counterclockwise]{rotating}   % Landscape page setup for large tables
\usepackage{multirow}		% Table styling
\usepackage{tabularx}		% Control Column width
\usepackage[flushleft]{threeparttable}	% Allows for three part tables with a specified notes section
\usepackage{threeparttablex}            % Lets threeparttable work with longtable

% Create new environments so endfloat can handle them
% \newenvironment{ltable}
%   {\begin{landscape}\begin{center}\begin{threeparttable}}
%   {\end{threeparttable}\end{center}\end{landscape}}
\newenvironment{lltable}{\begin{landscape}\begin{center}\begin{ThreePartTable}}{\end{ThreePartTable}\end{center}\end{landscape}}

% Enables adjusting longtable caption width to table width
% Solution found at http://golatex.de/longtable-mit-caption-so-breit-wie-die-tabelle-t15767.html
\makeatletter
\newcommand\LastLTentrywidth{1em}
\newlength\longtablewidth
\setlength{\longtablewidth}{1in}
\newcommand{\getlongtablewidth}{\begingroup \ifcsname LT@\roman{LT@tables}\endcsname \global\longtablewidth=0pt \renewcommand{\LT@entry}[2]{\global\advance\longtablewidth by ##2\relax\gdef\LastLTentrywidth{##2}}\@nameuse{LT@\roman{LT@tables}} \fi \endgroup}

% \setlength{\parindent}{0.5in}
% \setlength{\parskip}{0pt plus 0pt minus 0pt}

% \usepackage{etoolbox}
\makeatletter
\patchcmd{\HyOrg@maketitle}
  {\section{\normalfont\normalsize\abstractname}}
  {\section*{\normalfont\normalsize\abstractname}}
  {}{\typeout{Failed to patch abstract.}}
\patchcmd{\HyOrg@maketitle}
  {\section{\protect\normalfont{\@title}}}
  {\section*{\protect\normalfont{\@title}}}
  {}{\typeout{Failed to patch title.}}
\makeatother
\shorttitle{Stroop Analysis}
\keywords{Stroop, Reproducibilty\newline\indent Word count: X}
\DeclareDelayedFloatFlavor{ThreePartTable}{table}
\DeclareDelayedFloatFlavor{lltable}{table}
\DeclareDelayedFloatFlavor*{longtable}{table}
\makeatletter
\renewcommand{\efloat@iwrite}[1]{\immediate\expandafter\protected@write\csname efloat@post#1\endcsname{}}
\makeatother
\usepackage{csquotes}
\ifxetex
  % Load polyglossia as late as possible: uses bidi with RTL langages (e.g. Hebrew, Arabic)
  \usepackage{polyglossia}
  \setmainlanguage[]{english}
\else
  \usepackage[main=english]{babel}
% get rid of language-specific shorthands (see #6817):
\let\LanguageShortHands\languageshorthands
\def\languageshorthands#1{}
\fi
\ifluatex
  \usepackage{selnolig}  % disable illegal ligatures
\fi

\author{\phantom{0}}
\date{}


\affiliation{\phantom{0}}

\begin{document}

\begin{center}\rule{0.5\linewidth}{0.5pt}\end{center}

title : ``Testing rmd to pdf''
shorttitle : ``Stroop Analysis''

author:
- name : ``Matthew Crump''
affiliation : ``1''
corresponding : yes \# Define only one corresponding author
address : ``2900 Bedford Ave''
email : ``\href{mailto:mcrump@brooklyn.cuny.edu}{\nolinkurl{mcrump@brooklyn.cuny.edu}}''

affiliation:
- id : ``1''
institution : ``Brooklyn College of the City University of New York''

authornote: \textbar{}
Matthew J. C. Crump, Department of Psychology, Brooklyn College of the City University of New York.

abstract: \textbar{}
A reproduction of the analysis for Experiment 3 from Rosenbaum, Mama, and Algom (2017).

keywords : ``Stroop, Reproducibilty''
wordcount : ``X''

bibliography : {[}``r-references.bib,''``myreferences.bib''{]}

floatsintext : no
figurelist : no
tablelist : no
footnotelist : no
linenumbers : no
mask : no
draft : no

documentclass : ``apa6''
classoption : ``man''
output : papaja::apa6\_pdf
---

Problems

\begin{verbatim}
Should you do a linear regression on this data? Explain.
\end{verbatim}

Y \textless- c(10,9,8,7,6,5,4,3,2,1,2,3,4,5,6,7,8,9,10)
X \textless- c(1,2,3,4,5,6,7,8,9,10,11,12,13,14,15,16,17,18,19)

No, the data do not fit a linear regression. At first glance of the plot this seems to be the case- the result appears to be curvilinear and the values of X (the IV) do not permit predictions of Y (the DV). To be sure I ran the data through several of the processes from the lab12 text below and played around a bit to confirm that there is no linear correlation. Most importantly the correlation value from the cor.test = 0, shows that the relation between the variables X and Y is not linear. The p value of 1 also suggests that there is no statistically meaningful linear relation between the variables (that is, for this particular test/in the linear model).

\begin{Shaded}
\begin{Highlighting}[]
\NormalTok{Y }\OtherTok{\textless{}{-}} \FunctionTok{c}\NormalTok{(}\DecValTok{10}\NormalTok{,}\DecValTok{9}\NormalTok{,}\DecValTok{8}\NormalTok{,}\DecValTok{7}\NormalTok{,}\DecValTok{6}\NormalTok{,}\DecValTok{5}\NormalTok{,}\DecValTok{4}\NormalTok{,}\DecValTok{3}\NormalTok{,}\DecValTok{2}\NormalTok{,}\DecValTok{1}\NormalTok{,}\DecValTok{2}\NormalTok{,}\DecValTok{3}\NormalTok{,}\DecValTok{4}\NormalTok{,}\DecValTok{5}\NormalTok{,}\DecValTok{6}\NormalTok{,}\DecValTok{7}\NormalTok{,}\DecValTok{8}\NormalTok{,}\DecValTok{9}\NormalTok{,}\DecValTok{10}\NormalTok{)}
\NormalTok{X }\OtherTok{\textless{}{-}} \FunctionTok{c}\NormalTok{(}\DecValTok{1}\NormalTok{,}\DecValTok{2}\NormalTok{,}\DecValTok{3}\NormalTok{,}\DecValTok{4}\NormalTok{,}\DecValTok{5}\NormalTok{,}\DecValTok{6}\NormalTok{,}\DecValTok{7}\NormalTok{,}\DecValTok{8}\NormalTok{,}\DecValTok{9}\NormalTok{,}\DecValTok{10}\NormalTok{,}\DecValTok{11}\NormalTok{,}\DecValTok{12}\NormalTok{,}\DecValTok{13}\NormalTok{,}\DecValTok{14}\NormalTok{,}\DecValTok{15}\NormalTok{,}\DecValTok{16}\NormalTok{,}\DecValTok{17}\NormalTok{,}\DecValTok{18}\NormalTok{,}\DecValTok{19}\NormalTok{)}

\FunctionTok{plot}\NormalTok{(X,Y)}
\end{Highlighting}
\end{Shaded}

\includegraphics{CopyOflab12_files/figure-latex/unnamed-chunk-1-1.pdf}

\begin{Shaded}
\begin{Highlighting}[]
\NormalTok{some\_data }\OtherTok{\textless{}{-}} \FunctionTok{data.frame}\NormalTok{(}\AttributeTok{Y\_dv =}\NormalTok{ Y, }\AttributeTok{X\_iv=}\NormalTok{ X)}

\FunctionTok{lm}\NormalTok{(Y\_dv }\SpecialCharTok{\textasciitilde{}}\NormalTok{ X\_iv, }\AttributeTok{data =}\NormalTok{ some\_data)}
\end{Highlighting}
\end{Shaded}

\begin{verbatim}
## 
## Call:
## lm(formula = Y_dv ~ X_iv, data = some_data)
## 
## Coefficients:
## (Intercept)         X_iv  
##   5.737e+00    1.488e-16
\end{verbatim}

\begin{Shaded}
\begin{Highlighting}[]
\NormalTok{save\_model }\OtherTok{\textless{}{-}} \FunctionTok{lm}\NormalTok{(Y\_dv }\SpecialCharTok{\textasciitilde{}}\NormalTok{ X\_iv, }\AttributeTok{data=}\NormalTok{some\_data)}
\FunctionTok{summary}\NormalTok{(save\_model)}
\end{Highlighting}
\end{Shaded}

\begin{verbatim}
## 
## Call:
## lm(formula = Y_dv ~ X_iv, data = some_data)
## 
## Residuals:
##     Min      1Q  Median      3Q     Max 
## -4.7368 -2.2368  0.2632  2.2632  4.2632 
## 
## Coefficients:
##              Estimate Std. Error t value Pr(>|t|)    
## (Intercept) 5.737e+00  1.388e+00   4.132 0.000697 ***
## X_iv        1.488e-16  1.218e-01   0.000 1.000000    
## ---
## Signif. codes:  0 '***' 0.001 '**' 0.01 '*' 0.05 '.' 0.1 ' ' 1
## 
## Residual standard error: 2.907 on 17 degrees of freedom
## Multiple R-squared:  7.631e-31,  Adjusted R-squared:  -0.05882 
## F-statistic: 1.297e-29 on 1 and 17 DF,  p-value: 1
\end{verbatim}

\begin{Shaded}
\begin{Highlighting}[]
\FunctionTok{lm}\NormalTok{(}\AttributeTok{formula =}\NormalTok{ Y\_dv }\SpecialCharTok{\textasciitilde{}}\NormalTok{ X\_iv, }\AttributeTok{data =}\NormalTok{ some\_data)}
\end{Highlighting}
\end{Shaded}

\begin{verbatim}
## 
## Call:
## lm(formula = Y_dv ~ X_iv, data = some_data)
## 
## Coefficients:
## (Intercept)         X_iv  
##   5.737e+00    1.488e-16
\end{verbatim}

\begin{Shaded}
\begin{Highlighting}[]
\FunctionTok{cor.test}\NormalTok{(some\_data}\SpecialCharTok{$}\NormalTok{Y\_dv,some\_data}\SpecialCharTok{$}\NormalTok{X\_iv)}
\end{Highlighting}
\end{Shaded}

\begin{verbatim}
## 
##  Pearson's product-moment correlation
## 
## data:  some_data$Y_dv and some_data$X_iv
## t = 0, df = 17, p-value = 1
## alternative hypothesis: true correlation is not equal to 0
## 95 percent confidence interval:
##  -0.4542093  0.4542093
## sample estimates:
## cor 
##   0
\end{verbatim}

\begin{Shaded}
\begin{Highlighting}[]
\FunctionTok{summary}\NormalTok{(save\_model)}\SpecialCharTok{$}\NormalTok{r.squared}
\end{Highlighting}
\end{Shaded}

\begin{verbatim}
## [1] 7.631435e-31
\end{verbatim}

\begin{Shaded}
\begin{Highlighting}[]
\FunctionTok{sqrt}\NormalTok{(}\FunctionTok{summary}\NormalTok{(save\_model)}\SpecialCharTok{$}\NormalTok{r.squared)}
\end{Highlighting}
\end{Shaded}

\begin{verbatim}
## [1] 8.735808e-16
\end{verbatim}

\begin{Shaded}
\begin{Highlighting}[]
\FunctionTok{library}\NormalTok{(ggplot2)}

\FunctionTok{ggplot}\NormalTok{(some\_data, }\FunctionTok{aes}\NormalTok{(}\AttributeTok{x=}\NormalTok{X\_iv, }\AttributeTok{y=}\NormalTok{Y\_dv))}\SpecialCharTok{+}
  \FunctionTok{geom\_point}\NormalTok{()}\SpecialCharTok{+}
  \FunctionTok{geom\_smooth}\NormalTok{(}\AttributeTok{method=}\StringTok{\textquotesingle{}lm\textquotesingle{}}\NormalTok{, }\AttributeTok{se=}\ConstantTok{FALSE}\NormalTok{)}
\end{Highlighting}
\end{Shaded}

\includegraphics{CopyOflab12_files/figure-latex/unnamed-chunk-3-1.pdf}

\begin{Shaded}
\begin{Highlighting}[]
\NormalTok{some\_data2 }\OtherTok{\textless{}{-}} \FunctionTok{data.frame}\NormalTok{(}\AttributeTok{Y =} \FunctionTok{rnorm}\NormalTok{(}\DecValTok{4}\SpecialCharTok{*}\DecValTok{10}\NormalTok{,}\DecValTok{0}\NormalTok{,}\DecValTok{1}\NormalTok{),}
                         \AttributeTok{X =} \FunctionTok{rnorm}\NormalTok{(}\DecValTok{4}\SpecialCharTok{*}\DecValTok{10}\NormalTok{,}\DecValTok{0}\NormalTok{,}\DecValTok{1}\NormalTok{),}
                        \AttributeTok{set =} \FunctionTok{rep}\NormalTok{(}\DecValTok{1}\SpecialCharTok{:}\DecValTok{4}\NormalTok{,}\AttributeTok{each=}\DecValTok{10}\NormalTok{))}

\FunctionTok{ggplot}\NormalTok{(some\_data2, }\FunctionTok{aes}\NormalTok{(}\AttributeTok{x=}\NormalTok{X, }\AttributeTok{y=}\NormalTok{Y))}\SpecialCharTok{+}
  \FunctionTok{geom\_point}\NormalTok{()}\SpecialCharTok{+}
  \FunctionTok{geom\_smooth}\NormalTok{(}\AttributeTok{method=}\StringTok{\textquotesingle{}lm\textquotesingle{}}\NormalTok{, }\AttributeTok{se=}\ConstantTok{FALSE}\NormalTok{)}\SpecialCharTok{+}
  \FunctionTok{facet\_wrap}\NormalTok{(}\SpecialCharTok{\textasciitilde{}}\NormalTok{set)}
\end{Highlighting}
\end{Shaded}

\includegraphics{CopyOflab12_files/figure-latex/unnamed-chunk-4-1.pdf}

\begin{Shaded}
\begin{Highlighting}[]
\FunctionTok{library}\NormalTok{(ggpmisc)}
\NormalTok{formula }\OtherTok{\textless{}{-}}\NormalTok{ y }\SpecialCharTok{\textasciitilde{}}\NormalTok{ x}
\FunctionTok{ggplot}\NormalTok{(some\_data, }\FunctionTok{aes}\NormalTok{(}\AttributeTok{x=}\NormalTok{ X\_iv, }\AttributeTok{y =}\NormalTok{ Y\_dv)) }\SpecialCharTok{+}
  \FunctionTok{geom\_point}\NormalTok{() }\SpecialCharTok{+}
  \FunctionTok{stat\_smooth}\NormalTok{(}\AttributeTok{method =} \StringTok{"lm"}\NormalTok{, }\AttributeTok{formula =}\NormalTok{ formula) }\SpecialCharTok{+}
  \FunctionTok{stat\_poly\_eq}\NormalTok{(}\AttributeTok{formula =}\NormalTok{ formula, }\AttributeTok{parse =} \ConstantTok{TRUE}\NormalTok{)}
\end{Highlighting}
\end{Shaded}

\includegraphics{CopyOflab12_files/figure-latex/unnamed-chunk-4-2.pdf}

\begin{Shaded}
\begin{Highlighting}[]
\NormalTok{Y }\OtherTok{\textless{}{-}} \FunctionTok{c}\NormalTok{(}\DecValTok{10}\NormalTok{,}\DecValTok{9}\NormalTok{,}\DecValTok{8}\NormalTok{,}\DecValTok{7}\NormalTok{,}\DecValTok{6}\NormalTok{,}\DecValTok{5}\NormalTok{,}\DecValTok{4}\NormalTok{,}\DecValTok{3}\NormalTok{,}\DecValTok{2}\NormalTok{,}\DecValTok{1}\NormalTok{,}\DecValTok{2}\NormalTok{,}\DecValTok{3}\NormalTok{,}\DecValTok{4}\NormalTok{,}\DecValTok{5}\NormalTok{,}\DecValTok{6}\NormalTok{,}\DecValTok{7}\NormalTok{,}\DecValTok{8}\NormalTok{,}\DecValTok{9}\NormalTok{,}\DecValTok{10}\NormalTok{)}
\NormalTok{X }\OtherTok{\textless{}{-}} \FunctionTok{c}\NormalTok{(}\DecValTok{1}\NormalTok{,}\DecValTok{2}\NormalTok{,}\DecValTok{3}\NormalTok{,}\DecValTok{4}\NormalTok{,}\DecValTok{5}\NormalTok{,}\DecValTok{6}\NormalTok{,}\DecValTok{7}\NormalTok{,}\DecValTok{8}\NormalTok{,}\DecValTok{9}\NormalTok{,}\DecValTok{10}\NormalTok{,}\DecValTok{11}\NormalTok{,}\DecValTok{12}\NormalTok{,}\DecValTok{13}\NormalTok{,}\DecValTok{14}\NormalTok{,}\DecValTok{15}\NormalTok{,}\DecValTok{16}\NormalTok{,}\DecValTok{17}\NormalTok{,}\DecValTok{18}\NormalTok{,}\DecValTok{19}\NormalTok{)}

\FunctionTok{sort}\NormalTok{(Y, }\AttributeTok{decreasing =} \ConstantTok{TRUE}\NormalTok{)}
\end{Highlighting}
\end{Shaded}

\begin{verbatim}
##  [1] 10 10  9  9  8  8  7  7  6  6  5  5  4  4  3  3  2  2  1
\end{verbatim}

\begin{Shaded}
\begin{Highlighting}[]
\FunctionTok{sort}\NormalTok{(X, }\AttributeTok{decreasing =} \ConstantTok{TRUE}\NormalTok{)}
\end{Highlighting}
\end{Shaded}

\begin{verbatim}
##  [1] 19 18 17 16 15 14 13 12 11 10  9  8  7  6  5  4  3  2  1
\end{verbatim}

\begin{Shaded}
\begin{Highlighting}[]
\FunctionTok{plot}\NormalTok{(X,Y)}
\end{Highlighting}
\end{Shaded}

\includegraphics{CopyOflab12_files/figure-latex/unnamed-chunk-5-1.pdf}


\end{document}
