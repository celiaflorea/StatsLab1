% Options for packages loaded elsewhere
\PassOptionsToPackage{unicode}{hyperref}
\PassOptionsToPackage{hyphens}{url}
%
\documentclass[
  english,
  man]{apa6}
\usepackage{amsmath,amssymb}
\usepackage{lmodern}
\usepackage{ifxetex,ifluatex}
\ifnum 0\ifxetex 1\fi\ifluatex 1\fi=0 % if pdftex
  \usepackage[T1]{fontenc}
  \usepackage[utf8]{inputenc}
  \usepackage{textcomp} % provide euro and other symbols
\else % if luatex or xetex
  \usepackage{unicode-math}
  \defaultfontfeatures{Scale=MatchLowercase}
  \defaultfontfeatures[\rmfamily]{Ligatures=TeX,Scale=1}
\fi
% Use upquote if available, for straight quotes in verbatim environments
\IfFileExists{upquote.sty}{\usepackage{upquote}}{}
\IfFileExists{microtype.sty}{% use microtype if available
  \usepackage[]{microtype}
  \UseMicrotypeSet[protrusion]{basicmath} % disable protrusion for tt fonts
}{}
\makeatletter
\@ifundefined{KOMAClassName}{% if non-KOMA class
  \IfFileExists{parskip.sty}{%
    \usepackage{parskip}
  }{% else
    \setlength{\parindent}{0pt}
    \setlength{\parskip}{6pt plus 2pt minus 1pt}}
}{% if KOMA class
  \KOMAoptions{parskip=half}}
\makeatother
\usepackage{xcolor}
\IfFileExists{xurl.sty}{\usepackage{xurl}}{} % add URL line breaks if available
\IfFileExists{bookmark.sty}{\usepackage{bookmark}}{\usepackage{hyperref}}
\hypersetup{
  pdftitle={Reproducible analysis of Schroeder \& Epley, 2015: Do you come across as smarter when people read what you say or hear what you say},
  pdfauthor={Celia S. Florea1},
  pdflang={en-EN},
  pdfkeywords={keywords},
  hidelinks,
  pdfcreator={LaTeX via pandoc}}
\urlstyle{same} % disable monospaced font for URLs
\usepackage{graphicx}
\makeatletter
\def\maxwidth{\ifdim\Gin@nat@width>\linewidth\linewidth\else\Gin@nat@width\fi}
\def\maxheight{\ifdim\Gin@nat@height>\textheight\textheight\else\Gin@nat@height\fi}
\makeatother
% Scale images if necessary, so that they will not overflow the page
% margins by default, and it is still possible to overwrite the defaults
% using explicit options in \includegraphics[width, height, ...]{}
\setkeys{Gin}{width=\maxwidth,height=\maxheight,keepaspectratio}
% Set default figure placement to htbp
\makeatletter
\def\fps@figure{htbp}
\makeatother
\setlength{\emergencystretch}{3em} % prevent overfull lines
\providecommand{\tightlist}{%
  \setlength{\itemsep}{0pt}\setlength{\parskip}{0pt}}
\setcounter{secnumdepth}{-\maxdimen} % remove section numbering
% Make \paragraph and \subparagraph free-standing
\ifx\paragraph\undefined\else
  \let\oldparagraph\paragraph
  \renewcommand{\paragraph}[1]{\oldparagraph{#1}\mbox{}}
\fi
\ifx\subparagraph\undefined\else
  \let\oldsubparagraph\subparagraph
  \renewcommand{\subparagraph}[1]{\oldsubparagraph{#1}\mbox{}}
\fi
% Manuscript styling
\usepackage{upgreek}
\captionsetup{font=singlespacing,justification=justified}

% Table formatting
\usepackage{longtable}
\usepackage{lscape}
% \usepackage[counterclockwise]{rotating}   % Landscape page setup for large tables
\usepackage{multirow}		% Table styling
\usepackage{tabularx}		% Control Column width
\usepackage[flushleft]{threeparttable}	% Allows for three part tables with a specified notes section
\usepackage{threeparttablex}            % Lets threeparttable work with longtable

% Create new environments so endfloat can handle them
% \newenvironment{ltable}
%   {\begin{landscape}\begin{center}\begin{threeparttable}}
%   {\end{threeparttable}\end{center}\end{landscape}}
\newenvironment{lltable}{\begin{landscape}\begin{center}\begin{ThreePartTable}}{\end{ThreePartTable}\end{center}\end{landscape}}

% Enables adjusting longtable caption width to table width
% Solution found at http://golatex.de/longtable-mit-caption-so-breit-wie-die-tabelle-t15767.html
\makeatletter
\newcommand\LastLTentrywidth{1em}
\newlength\longtablewidth
\setlength{\longtablewidth}{1in}
\newcommand{\getlongtablewidth}{\begingroup \ifcsname LT@\roman{LT@tables}\endcsname \global\longtablewidth=0pt \renewcommand{\LT@entry}[2]{\global\advance\longtablewidth by ##2\relax\gdef\LastLTentrywidth{##2}}\@nameuse{LT@\roman{LT@tables}} \fi \endgroup}

% \setlength{\parindent}{0.5in}
% \setlength{\parskip}{0pt plus 0pt minus 0pt}

% \usepackage{etoolbox}
\makeatletter
\patchcmd{\HyOrg@maketitle}
  {\section{\normalfont\normalsize\abstractname}}
  {\section*{\normalfont\normalsize\abstractname}}
  {}{\typeout{Failed to patch abstract.}}
\patchcmd{\HyOrg@maketitle}
  {\section{\protect\normalfont{\@title}}}
  {\section*{\protect\normalfont{\@title}}}
  {}{\typeout{Failed to patch title.}}
\makeatother
\shorttitle{Semester Project}
\keywords{keywords\newline\indent Word count: X}
\DeclareDelayedFloatFlavor{ThreePartTable}{table}
\DeclareDelayedFloatFlavor{lltable}{table}
\DeclareDelayedFloatFlavor*{longtable}{table}
\makeatletter
\renewcommand{\efloat@iwrite}[1]{\immediate\expandafter\protected@write\csname efloat@post#1\endcsname{}}
\makeatother
\usepackage{lineno}

\linenumbers
\usepackage{csquotes}
\ifxetex
  % Load polyglossia as late as possible: uses bidi with RTL langages (e.g. Hebrew, Arabic)
  \usepackage{polyglossia}
  \setmainlanguage[]{english}
\else
  \usepackage[main=english]{babel}
% get rid of language-specific shorthands (see #6817):
\let\LanguageShortHands\languageshorthands
\def\languageshorthands#1{}
\fi
\ifluatex
  \usepackage{selnolig}  % disable illegal ligatures
\fi
\newlength{\cslhangindent}
\setlength{\cslhangindent}{1.5em}
\newlength{\csllabelwidth}
\setlength{\csllabelwidth}{3em}
\newenvironment{CSLReferences}[2] % #1 hanging-ident, #2 entry spacing
 {% don't indent paragraphs
  \setlength{\parindent}{0pt}
  % turn on hanging indent if param 1 is 1
  \ifodd #1 \everypar{\setlength{\hangindent}{\cslhangindent}}\ignorespaces\fi
  % set entry spacing
  \ifnum #2 > 0
  \setlength{\parskip}{#2\baselineskip}
  \fi
 }%
 {}
\usepackage{calc}
\newcommand{\CSLBlock}[1]{#1\hfill\break}
\newcommand{\CSLLeftMargin}[1]{\parbox[t]{\csllabelwidth}{#1}}
\newcommand{\CSLRightInline}[1]{\parbox[t]{\linewidth - \csllabelwidth}{#1}\break}
\newcommand{\CSLIndent}[1]{\hspace{\cslhangindent}#1}

\title{Reproducible analysis of Schroeder \& Epley, 2015: Do you come across as smarter when people read what you say or hear what you say}
\author{Celia S. Florea\textsuperscript{1}}
\date{}


\authornote{

Celia S. Florea, Department of Psychology, Brooklyn College of the City University of New York.

Correspondence concerning this article should be addressed to Celia S. Florea, 2900 Bedford Avenue, Brooklyn NY. E-mail: \href{mailto:celia.florea99@bcmail.cuny.edu}{\nolinkurl{celia.florea99@bcmail.cuny.edu}}

}

\affiliation{\vspace{0.5cm}\textsuperscript{1} Brooklyn College of the City University of New York}

\abstract{
One or two sentences providing a \textbf{basic introduction} to the field, comprehensible to a scientist in any discipline.

Two to three sentences of \textbf{more detailed background}, comprehensible to scientists in related disciplines.

One sentence clearly stating the \textbf{general problem} being addressed by this particular study.

One sentence summarizing the main result (with the words ``\textbf{here we show}'' or their equivalent).

Two or three sentences explaining what the \textbf{main result} reveals in direct comparison to what was thought to be the case previously, or how the main result adds to previous knowledge.

One or two sentences to put the results into a more \textbf{general context}.

Two or three sentences to provide a \textbf{broader perspective}, readily comprehensible to a scientist in any discipline.
}



\begin{document}
\maketitle

\hypertarget{method}{%
\section{Method}\label{method}}

This report reproduces the analysis from experiment 4 in Schroeder \& Epley (2015). The data were downloaded from the ``open data'' folder for this class, from the file named ``SchroederEpley2015data.csv.'' In experiment 4 Schroeder \& Epley (2015) replicated the results of experiments 1-3 but here enlisted professional recruiters as participants to improve the ecological validity of their experiment.

\hypertarget{participants}{%
\subsection{Participants}\label{participants}}

The participants N=39 (mean age=30.85 years, SD= 6.24, 30 females) in experiment 4 were professional recruiters from fortune 500 companies who had agreed to evaluate potential candidates at the University of Chicago Booth School of Business. The experimenters reached out to 66 recruiters who had attended such a jobs conference at the University of Chicago via email to request their participation in a survey. Of the 66 recruiters contacted, 39 responded and agreed to participate.

\hypertarget{materials}{%
\subsection{Materials}\label{materials}}

The stimuli which were developed for experiment 1, of which a subset were used here in experiment 4, were created from video recordings of MBA students spoken elevator pitches made for potential employers. It was predicted that evaluators would respond more positively to the pitches that they heard rather than those they read, as this would make the candidates appear more thoughtful and intelligent.

The survey then asked participants to rate each potential candidate on 3 dimensions: the candidate's competence (as compared to the average candidate for a similar position), the candidate's thoughtfulness, and the candidates' intelligence. The recruiters were then asked to rate their general impressions of the candidates with questions that probed how much they liked the candidate, how positive and negative their overall impressions were and whether or not they would opt to hire the candidate.

\hypertarget{procedure}{%
\subsection{Procedure}\label{procedure}}

Participants responded to an online survey and were randomly assigned to either listen to recordings of spoken pitches (audio condition) or the same pitch in text (transcript condition) and answered survey questions. The materials were the same as experiment 1 (except that there was no video condition in experiment 4). The survey questions were rated on a likert type scale from 0-10 (e.g.~0 = much less thoughtful, 10 = much more thoughtful).

The recruiters ratings of the job candidates pitches were collapsed into into composite measures of intellect (cronbach's alpha = .92) and general impressions (cronbach's alpha = .93).

\hypertarget{data-analysis}{%
\subsection{Data analysis}\label{data-analysis}}

We used R {[}Version 4.0.2; R Core Team (2020){]} and the R-packages \emph{crayon} {[}Version 1.4.0; Csárdi (2017){]}, \emph{csvread} {[}Version 1.2.1; Izrailev (2018){]}, \emph{data.table} {[}Version 1.13.6; Dowle and Srinivasan (2020){]}, \emph{dplyr} {[}Version 1.0.3; Wickham, François, Henry, and Müller (2021){]}, \emph{ggplot2} {[}Version 3.3.3; Wickham (2016){]}, \emph{ggpmisc} {[}Version 0.3.8.1; Aphalo (2021){]}, \emph{kableExtra} {[}Version 1.3.1; Zhu (2020){]}, \emph{papaja} {[}Version 0.1.0.9997; Aust and Barth (2020){]}, \emph{readr} {[}Version 1.4.0; Wickham, Hester, and Francois (2018){]}, \emph{tibble} {[}Version 3.0.6; Müller and Wickham (2021){]}, \emph{tidyr} {[}Version 1.1.2; Wickham (2020){]}, and \emph{tinytex} {[}Version 0.29; Xie (2019){]} for all our analyses.

\hypertarget{results}{%
\section{Results}\label{results}}

\begin{verbatim}
##    CONDITION Intellect_Rating
## 1 transcript         3.648148
## 2      audio         5.634921
\end{verbatim}

\begin{verbatim}
##    CONDITION Impression_Rating
## 1 transcript          4.074074
## 2      audio          5.968254
\end{verbatim}

\begin{verbatim}
##    CONDITION Hire_Rating
## 1 transcript    2.888889
## 2      audio    4.714286
\end{verbatim}

\begin{verbatim}
## 
##  Two Sample t-test
## 
## data:  Intellect_Rating by CONDITION
## t = -3.5259, df = 37, p-value = 0.001144
## alternative hypothesis: true difference in means is not equal to 0
## 95 percent confidence interval:
##  -3.1284798 -0.8450652
## sample estimates:
## mean in group transcript      mean in group audio 
##                 3.648148                 5.634921
\end{verbatim}

\begin{verbatim}
## 
##  Two Sample t-test
## 
## data:  Impression_Rating by CONDITION
## t = -2.8508, df = 37, p-value = 0.007091
## alternative hypothesis: true difference in means is not equal to 0
## 95 percent confidence interval:
##  -3.2404752 -0.5478846
## sample estimates:
## mean in group transcript      mean in group audio 
##                 4.074074                 5.968254
\end{verbatim}

\begin{verbatim}
## 
##  Two Sample t-test
## 
## data:  Hire_Rating by CONDITION
## t = -2.6201, df = 37, p-value = 0.01267
## alternative hypothesis: true difference in means is not equal to 0
## 95 percent confidence interval:
##  -3.2370242 -0.4137694
## sample estimates:
## mean in group transcript      mean in group audio 
##                 2.888889                 4.714286
\end{verbatim}

\includegraphics{semester-project.csf_files/figure-latex/unnamed-chunk-2-1.pdf}
\includegraphics{semester-project.csf_files/figure-latex/unnamed-chunk-3-1.pdf}
\includegraphics{semester-project.csf_files/figure-latex/unnamed-chunk-4-1.pdf}

\includegraphics{semester-project.csf_files/figure-latex/unnamed-chunk-5-1.pdf}

\hypertarget{discussion}{%
\section{Discussion}\label{discussion}}

\newpage

\hypertarget{references}{%
\section{References}\label{references}}

Schroeder, J., \& Epley, N. (2015). The Sound of Intellect: Speech Reveals a Thoughtful Mind, Increasing a Job Candidate's Appeal. Psychological Science, 1, 15.

\begingroup
\setlength{\parindent}{-0.5in}
\setlength{\leftskip}{0.5in}

\hypertarget{refs}{}
\begin{CSLReferences}{1}{0}
\leavevmode\hypertarget{ref-R-ggpmisc}{}%
Aphalo, P. J. (2021). \emph{Ggpmisc: Miscellaneous extensions to 'ggplot2'}. Retrieved from \url{https://CRAN.R-project.org/package=ggpmisc}

\leavevmode\hypertarget{ref-R-papaja}{}%
Aust, F., \& Barth, M. (2020). \emph{{papaja}: {Create} {APA} manuscripts with {R Markdown}}. Retrieved from \url{https://github.com/crsh/papaja}

\leavevmode\hypertarget{ref-R-crayon}{}%
Csárdi, G. (2017). \emph{Crayon: Colored terminal output}. Retrieved from \url{https://CRAN.R-project.org/package=crayon}

\leavevmode\hypertarget{ref-R-data.table}{}%
Dowle, M., \& Srinivasan, A. (2020). \emph{Data.table: Extension of `data.frame`}. Retrieved from \url{https://CRAN.R-project.org/package=data.table}

\leavevmode\hypertarget{ref-R-csvread}{}%
Izrailev, S. (2018). \emph{Csvread: Fast specialized CSV file loader}. Retrieved from \url{https://CRAN.R-project.org/package=csvread}

\leavevmode\hypertarget{ref-R-tibble}{}%
Müller, K., \& Wickham, H. (2021). \emph{Tibble: Simple data frames}. Retrieved from \url{https://CRAN.R-project.org/package=tibble}

\leavevmode\hypertarget{ref-R-base}{}%
R Core Team. (2020). \emph{R: A language and environment for statistical computing}. Vienna, Austria: R Foundation for Statistical Computing. Retrieved from \url{https://www.R-project.org/}

\leavevmode\hypertarget{ref-R-ggplot2}{}%
Wickham, H. (2016). \emph{ggplot2: Elegant graphics for data analysis}. Springer-Verlag New York. Retrieved from \url{https://ggplot2.tidyverse.org}

\leavevmode\hypertarget{ref-R-tidyr}{}%
Wickham, H. (2020). \emph{Tidyr: Tidy messy data}. Retrieved from \url{https://CRAN.R-project.org/package=tidyr}

\leavevmode\hypertarget{ref-R-dplyr}{}%
Wickham, H., François, R., Henry, L., \& Müller, K. (2021). \emph{Dplyr: A grammar of data manipulation}. Retrieved from \url{https://CRAN.R-project.org/package=dplyr}

\leavevmode\hypertarget{ref-R-readr}{}%
Wickham, H., Hester, J., \& Francois, R. (2018). \emph{Readr: Read rectangular text data}. Retrieved from \url{https://CRAN.R-project.org/package=readr}

\leavevmode\hypertarget{ref-R-tinytex}{}%
Xie, Y. (2019). TinyTeX: A lightweight, cross-platform, and easy-to-maintain LaTeX distribution based on TeX live. \emph{TUGboat}, (1), 30--32. Retrieved from \url{http://tug.org/TUGboat/Contents/contents40-1.html}

\leavevmode\hypertarget{ref-R-kableExtra}{}%
Zhu, H. (2020). \emph{kableExtra: Construct complex table with 'kable' and pipe syntax}. Retrieved from \url{https://CRAN.R-project.org/package=kableExtra}

\end{CSLReferences}

\endgroup


\end{document}
